% Open Logic Project
%
% driver file open-logic-sample.tex to produce text on letter-size paper
% with standard layout and margins

% We use the memoir class for maximal flexibility of layout, but any
% class will do

\documentclass[a4paper]{memoir}

% \olpath has to point to the location of the OLP main
% directory/folder.  We're compiling from subdirectory courses/sample,
% so the main directory is two levels up.
\newcommand{\olpath}{../../}

% load all the Open Logic definitions. This will also load the
% local definitions in open-logic-sample-config.sty
\input{\olpath/sty/open-logic.sty}

% we want all the problems deferred to the end
\input{\olpath/sty/open-logic-defer.sty}

% let's set the whole thing in Palatino, with Helvetica for
% sans-serif, and spread the lines a bit to make the text more
% readable

% \usepackage{mathpazo}
% \usepackage[scaled=0.95]{helvet}
% \linespread{1.05}

\usepackage{libertine}
\usepackage[libertine]{newtxmath}

\usepackage{microtype}

\begin{document}

% First we make a titlepage

\begin{titlingpage}
\begin{raggedleft}
\fontsize{52pt}{2em}\selectfont\bfseries\sffamily
Sets\\[.5ex]
Relations\\[.5ex] 
Functions
\vskip 4ex
\normalfont\Huge\textbf{\href{http://openlogicproject.org/}{Open Logic Project}}

\end{raggedleft}

\vfill

% oluselicense generates a license mark that a) licenses the result
% under a CC-BY licence and b) acknowledges the original source (the
% OLP).  Acknowledgment of the source is a requirement under the
% conditions of the CC-BY license used by the OLP, but you are not
% required to license the product itself under CC-BY.
\begin{minipage}[b]{.9cm}
    \includegraphics[width=.9cm]{\olpath/assets/logos/by}
    \includegraphics[width=.9cm]{\olpath/assets/logos/cc}
    \includegraphics[width=.9cm]{\olpath/assets/logos/remix}
    \end{minipage}
    \hspace{.3cm}
    \begin{minipage}[b]{6.5cm}
    \ollicensefont
    \textit{Sets, Relations, Functions} is licensed under a
    \href{http://creativecommons.org/licenses/by/4.0/}{Creative Commons
      Attribution 4.0 International License}. It is based on
    \textit{\href{https://github.com/OpenLogicProject/OpenLogic}{The Open
        Logic Text}} by the \href{http://openlogicproject.org/}{Open Logic
      Project}, used under a
    \href{http://creativecommons.org/licenses/by/4.0/}{Creative Commons
      Attribution 4.0 International License}, and \textit{\href{http://people.ds.cam.ac.uk/tecb2/metatheory.shtml}{Metatheory}} by Tim Button, also under a \href{http://creativecommons.org/licenses/by/4.0/}{Creative Commons
      Attribution 4.0 International License}.
    \end{minipage}
% \oluselicense
% % Title of this version of the OLT with link to source
% {\href{https://github.com/OpenLogicProject/OpenLogic/tree/master/courses/sample}{\textit{Sample Logic Text}}}
% % Author of this version
% {\href{http://openlogicproject.org/}{OLP}}
\end{titlingpage}

\frontmatter
\pagestyle{ruled}

\tableofcontents*

\mainmatter

% olimport includes an entire part

\olimport*[sets-functions-relations]{sets-functions-relations}
% \olimport[sets]{sets}

% \olimport[relations]{relations}

% \olimport[functions]{functions}

% you can also import individual chapters, but then don't forget to
% include part headings

\OLEndPartHook

% \stopproblems

% % Ok, that's it. Now for the appendices

% \appendix

% \olimport*[methods]{methods}

% \olimport*[history/biographies]{biographies}

% % now typeset all the problems as an appendix. If you want problems at
% % the end of each chapter, delete this part and put
% % \problemsperchapter in the preamble

% \chapter{Problems}

% \printproblems

\backmatter

% If you include any chapters from the history part, you have to print
% the Photo Credits. 

% \photocredits

% Include the bibliography

\bibliographystyle{\olpath/bib/natbib-oup}
\bibliography{\olpath/bib/open-logic}

\end{document}

