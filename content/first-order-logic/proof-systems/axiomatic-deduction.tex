% Part: first-order-logic
% Chapter: proof-systems
% Section: axiomatic-deduction

\documentclass[../../../include/open-logic-section]{subfiles}

\begin{document}

\iftag{FOL}
      {\olfileid{fol}{prf}{axd}}
      {\olfileid{pl}{prf}{axd}}

\olsection{Axiomatic \usetoken{P}{derivation}}

Axiomatic !!{derivation}s are the oldest and simplest logical
!!{derivation} systems. Its !!{derivation}s are simply sequences of
!!{sentence}s.  A sequence of !!{sentence}s counts as a correct
!!{derivation} if every !!{sentence}~$!A$ in it satisfies one of the
following conditions:
\begin{enumerate}
\item $!A$ is an axiom, or
\item $!A$ is !!a{element} of a given set~$\Gamma$ of !!{sentence}s, or
\item $!A$ is justified by a rule of inference.
\end{enumerate}
To be an axiom, $!A$ has to have the form of one of a number of fixed
!!{sentence} schemas. There are many sets of axiom schemas that
provide a satisfactory (sound and complete) !!{derivation} system for
first-order logic. Some are organized according to the connectives
they govern, e.g., the schemas
\[
!A \lif (!B \lif !A) \qquad !B \lif (!B \lor !C) \qquad (!B \land !C) \lif !B
\]
are common axioms that govern $\lif$, $\lor$ and~$\land$. Some axiom systems
aim at a minimal number of axioms. Depending on the connectives that
are taken as primitives, it is even possible to find axiom systems
that consist of a single axiom.

A rule of inference is a conditional statement that gives a sufficient
condition for !!a{sentence} in !!a{derivation} to be justified. Modus
ponens is one very common such rule: it says that if $!A$ and $!A \lif
!B$ are already justified, then $!B$ is justified. This means that a
line in !!a{derivation} containing the !!{sentence}~$!B$ is justified,
provided that both $!A$ and $!A \lif !B$ (for some !!{sentence}~$!A$)
appear in the !!{derivation} before~$!B$.

The $\Proves$ relation based on axiomatic !!{derivation}s is defined
as follows: $\Gamma \Proves !A$ iff there is !!a{derivation} with the
!!{sentence}~$!A$ as its last formula (and $\Gamma$ is taken as the
set of !!{sentence}s in that derivation which are justified by~(2) above).  $!A$
is a theorem if~$!A$ has !!a{derivation} where~$\Gamma$ is empty,
i.e., every !!{sentence} in the derivation is justfied either by (1)
or~(3). For instance, here is !!a{derivation} that shows that $\Proves
!A  \lif (!B \lif (!B \lor !A))$:
\begin{derivation}
  1. & $!B \lif (!B \lor !A)$ \\
  2. & $(!B \lif (!B \lor !A)) \lif (!A  \lif (!B \lif (!B \lor !A)))$\\
  3. & $!A  \lif (!B \lif (!B \lor !A))$
\end{derivation}
The !!{sentence} on line~1 is of the form of the axiom $!A \lif (!A
\lor !B)$ (with the roles of $!A$ and $!B$ reversed). The sentence on
line~2 is of the form of the axiom $!A \lif (!B \lif !A)$. Thus, both
lines are justified. Line~3 is justified by modus ponens: if we
abbreviate it as $!D$, then line~2 has the form $!C \lif !D$, where
$!C$ is $!B \lif (!B \lor !A)$, i.e., line~1.

A set $\Gamma$ is inconsistent if $\Gamma \Proves \lfalse$. A complete
axiom system will also prove that $\lfalse \lif !A$ for any~$!A$, and
so if $\Gamma$ is inconsistent, then $\Gamma \Proves !A$ for any~$!A$.

Systems of axiomatic !!{derivation}s for logic were first given by
Gottlob Frege in his 1879 \emph{Begriffsschrift}, which for this
reason is often considered the first work of modern logic. They were
perfected in Alfred North Whitehead and Bertrand Russell's
\emph{Principia Mathematica} and by David Hilbert and his students in
the 1920s. They are thus often called ``Frege systems'' or ``Hilbert
systems.'' They are very versatile in that it is often easy to find
an axiomatic system for a logic. Because !!{derivation}s have a very
simple structure and only one or two inference rules, it is also
relatively easy to prove things \emph{about} them. However, they are
very hard to use in practice, i.e., it is difficult to find and write
proofs.

\end{document}
