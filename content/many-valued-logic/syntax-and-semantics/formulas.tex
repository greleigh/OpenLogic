% Part: many-valued-logic
% Chapter: syntax-and-semantics
% Section: formulas

\documentclass[../../../include/open-logic-section]{subfiles}

\begin{document}

\olfileid{mvl}{syn}{fml}

\olsection{\usetoken{P}{formula}}

\begin{defn}[Formula]
\ollabel{defn:formulas}
The set~$\Frm[L]$ of \emph{!!{formula}s} of a propositional
language~$\Lang L$ is defined inductively as follows:
\begin{enumerate}
\item Every !!{propositional variable}~$\Obj p_i$ is an atomic
  !!{formula}.
\item Every $0$-place connective (propositional constant) of~$\Lang L$
is an atomic !!{formula}.
\item If $\star$ is an $n$-place connective of~$\Lang L$, and $!A_1$,
\dots, $!A_n$ are !!{formula}s, then $\star(!A_1, \dots, !A_n)$ is
  !!a{formula}.
\tagitem{limitClause}{Nothing else is !!a{formula}.}{}
\end{enumerate}
If $\star$ is $1$-place, then $\star(!A_1)$ will often be written
simply as $\star !A_1$. If $\star$ is $2$-place $\star(!A_1,!A_2)$
will often be written as $(!A_1 \star !A_2)$. 
\end{defn}

As usual, we will often silently leave out the outermost parentheses.

\begin{ex}
  In the standard language~$\Lang{L_0}$, $\Obj p_1 \lif (\Obj p_1
  \land \lnot \Obj p_2)$ is a formula. In the language of product
  logic, it would be written instead as $\Obj p_1 \lif (\Obj p_1 \odot
  \lnot \Obj p_2)$.  If we add the $1$-place $\triangle$ to the
  language, we would also have formulas such as $\triangle (\Obj p_1
  \land \Obj p_2) \lif (\triangle \Obj p_1 \land \triangle \Obj p_2)$.
\end{ex}

\end{document}
